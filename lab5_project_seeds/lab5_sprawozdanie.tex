\documentclass[a4paper,12pt]{article}

\usepackage[T1]{fontenc}
\usepackage[utf8]{inputenc}
\usepackage{graphicx}
\usepackage{xcolor}
\renewcommand\familydefault{\sfdefault}
\usepackage{tgheros}
\usepackage[defaultmono]{droidmono}
\usepackage{amsmath,amssymb,amsthm,textcomp}
\usepackage{enumerate}
\usepackage{multicol}
\usepackage{tikz}
\usepackage{float}
\usepackage{geometry}
\geometry{left=25mm,right=25mm,%
bindingoffset=0mm, top=20mm,bottom=20mm}
\linespread{1.3}
\newcommand{\linia}{\rule{\linewidth}{0.5pt}}

% custom theorems if needed
\newtheoremstyle{mytheor}
    {1ex}{1ex}{\normalfont}{0pt}{\scshape}{.}{1ex}
    {{\thmname{#1 }}{\thmnumber{#2}}{\thmnote{ (#3)}}}

\theoremstyle{mytheor}
\newtheorem{defi}{Definition}

% my own titles
\makeatletter
\renewcommand{\maketitle}{
\begin{center}
\vspace{2ex}
{\huge \textsc{\@title}}
\vspace{1ex}
\\
\linia\\
\@author \hfill \@date
\vspace{4ex}
\end{center}
}
\makeatother
%%%

% custom footers and headers
\usepackage{fancyhdr}
\usepackage{hyperref}
\usepackage{listings}
\pagestyle{fancy}
\lhead{}
\chead{}
\rhead{}
\lfoot{Sztuczna Inteligencja, Sprawozdanie nr 5}
\cfoot{}
\rfoot{Strona \thepage}
\renewcommand{\headrulewidth}{0pt}
\renewcommand{\footrulewidth}{0pt}
\usepackage[tablename=Tablica]{caption}
%

%%%----------%%%----------%%%----------%%%----------%%%

\begin{document}

\title{SI - SPRAWOZDANIE LAB nr 5}

\author{Maciej Budzowski, dzienne, grupa L5}

\date{09/05/2020}

\maketitle
Link do pliku .tex na platformie OverLeaf: \textcolor{red}{\href{https://www.overleaf.com/read/gtjhbdddtygc}{OVERLEAF}}\\
\section*{Wstęp}

Za zadanie miałem wytrenować sieć neuronową dla różnej liczby neuronowej w warstwie ukrytej oraz zbadać wpływ liczby neuronowej na otrzymane rezultaty.

\section*{Użyty zbiór danych}

Do wykonania zadania użyłem zbioru \textbf{\emph{seeds}}, linki do źródeł zamieszczam poniżej:\\
\textcolor{red}{\href{https://archive.ics.uci.edu/ml/machine-learning-databases/00236/seeds_dataset.txt}{DANE (PLIK)}} (link prowadzi do oryginalnego pliku)\\
\textcolor{red}{\href{https://archive.ics.uci.edu/ml/datasets/seeds}{DANE (STRONA)}} (link prowadzi do strony datasetu)\\

\textbf{Opis/temat zbioru:} Pomiary właściwości geometrycznych ziaren należących do trzech różnych odmian pszenicy. Zastosowano miękką technikę rentgenowską i pakiet ZIARNA do skonstruowania wszystkich siedmiu atrybutów o wartościach rzeczywistych.\\

\textbf{Autorzy:}\\
Małgorzata Charytanowicz, Jerzy Niewczas\\
Institute of Mathematics and Computer Science,\\
The John Paul II Catholic University of Lublin, Konstantynów 1 H\\
PL 20-708 Lublin, Poland\\

\textbf{Parametry użytego zbioru:}\\
 - liczba obserwacji - 210\\
 - liczba cech - 7 (+1 - klasa)\\
 - liczba klas - 3 (1, 2, 3 - odpowiednio "Kama", "Rosa", "Canadian")\\
 - liczba elementów na klasę - "po równo" - 70 elementów dla każdej z klas\\

\textbf{Rodzaje cech w zbiorze:}\\
 - area (A) (type - real)\\
 - perimeter (P) (type - real)\\
 - compactness (C = 4*pi*A/P2) (type - real)\\
 - length of kernel (type - real)\\
 - width of kernel (type - real)\\
 - asymmetry coefficient (type - real)\\
 - length of kernel groove (type - real)\\
 - class (nazwa klasy)\\

\section*{Przygotowanie danych}
\linia\\
 W ramach przygotowania danych do pracy musiałem podmienić nazwy klas (w oryginalnym pliku zostały użyte liczby - 1, 2, 3), użyłem w tym celu VSCode oraz narzędzia regex. W skrypcie jako komentarz również pojawia się informacja na ten temat.\\
 
 \textbf{Link do oryginalnych danych:} \textcolor{red}{\href{https://github.com/bvdzynski/up_ai-lab/blob/master/lab5_project_seeds/data/seeds_dataset_original.txt}{ORYGINALNE}}\\
 
 \textbf{Link do przygotowanych danych:} \textcolor{red}{\href{https://github.com/bvdzynski/up_ai-lab/blob/master/lab5_project_seeds/data/seeds_dataset_changed.data}{PRZYGOTOWANE}}\\
 
\section*{Kod}
\linia\\
\textbf{Link do projektu ze skryptem:} \textcolor{red}{\href{https://github.com/bvdzynski/up_ai-lab/tree/master/lab5_project_seeds}{GITHUB}}\\

Klasyfikacji dokonałem przy użyciu skryptu języka R.
Opiera się on na kodzie z odbytych ćwiczeń, jednak na potrzeby zadania zaszła potrzeba modyfikacji.\\

Część właściwego treningu i wypisywania wyników zamknąłem w funkcji, która zostaje wywołana w pętli wraz z odpowiednią wartością \textbf{\emph{neuralCount}} (ilości neutronów). Pozwoliło to na automatyzację procesu. (Podczas prób zauważyłem, że zwiększenie ilości neuronów znacząco wydłuża czas skryptu.)\\

Pozwoliłem sobie również na zwiększenie parametru \textbf{\emph{threshold}} na 0.1 (domyślna wartość 0.01)

\begin{lstlisting}[language=R]
nn <- neuralnet(Kama + Rosa + Canadian ~
                area + perimeter + compactness + 
                kernel_length + kernel_width + 
                asymmetry_coefficient + length_kernel_groove,
              data = neuralNet_seedsTrain,
              hidden = c(neuralCount),
              stepmax = 1e+06,
              threshold = 0.1
)
\end{lstlisting}

\section*{Wyniki}
\linia\\
*Zbiory treningowe i walidacyjne - każdy z nich stanowi połowę całości rekordów.\\

\textbf{Liczba neuronów: 5}\\

\begin{table}[H]
\begin{tabular}{ccccc}
prediction & Canadian & Kama & Rosa  &  \\
Canadian & \textbf{36} & 1 & 0 &  \\
Kama & 4 & \textbf{30} & 1 &  \\
Rosa & 0 & 0 & \textbf{33} &  \\
&&&& 
\end{tabular}
\caption{\textit{macierz pomyłek dla 5 neuronów w warstwie ukrytej}}
\label{tab:1}
\end{table}
\begin{table}[H]
\begin{tabular}{ccccc}
prediction & TPR (True Positive Rate) &  \\
Canadian & 36/40 (90\%) & \\
Kama & 30/31 (ok. 97\%) & \\
Rosa & 33/34 (ok. 97\%) & \\
&&
\end{tabular}
\caption{\textit{True Positive Rate dla 5 neuronów w warstwie ukrytej}}
\label{tab:2}
\end{table}
\textbf{Recognition Rate: 0.9428571}\\
\linia\\

\textbf{Liczba neuronów: 10}\\

\begin{table}[H]
\begin{tabular}{ccccc}
prediction & Canadian & Kama & Rosa  &  \\
Canadian & \textbf{35} & 1 & 0 &  \\
Kama & 5 & \textbf{30} & 0 &  \\
Rosa & 0 & 0 & \textbf{34} &  \\
&&&& 
\end{tabular}
\caption{\textit{macierz pomyłek dla 10 neuronów w warstwie ukrytej}}
\label{tab:3}
\end{table}
\begin{table}[H]
\begin{tabular}{ccccc}
prediction & TPR (True Positive Rate) &  \\
Canadian & 35/40 (87,5\%) & \\
Kama & 30/31 (ok. 97\%) & \\
Rosa & 34/34 (100\%) & \\
&&
\end{tabular}
\caption{\textit{True Positive Rate dla 10 neuronów w warstwie ukrytej}}
\label{tab:4}
\end{table}
\textbf{Recognition Rate: 0.9428571}\\
\linia\\

\textbf{Liczba neuronów: 15}\\

\begin{table}[H]
\begin{tabular}{ccccc}
prediction & Canadian & Kama & Rosa  &  \\
Canadian & \textbf{36} & 1 & 0 &  \\
Kama & 4 & \textbf{30} & 0 &  \\
Rosa & 0 & 0 & \textbf{34} &  \\
&&&& 
\end{tabular}
\caption{\textit{macierz pomyłek dla 15 neuronów w warstwie ukrytej}}
\label{tab:5}
\end{table}
\begin{table}[H]
\begin{tabular}{ccccc}
prediction & TPR (True Positive Rate) &  \\
Canadian & 36/40 (90\%) & \\
Kama & 30/31 (ok. 97\%) & \\
Rosa & 34/34 (100\%) & \\
&&
\end{tabular}
\caption{\textit{True Positive Rate dla 15 neuronów w warstwie ukrytej}}
\label{tab:6}
\end{table}
\textbf{Recognition Rate: 0.952381}\\
\linia\\

\textbf{Liczba neuronów: 20}\\

\begin{table}[H]
\begin{tabular}{ccccc}
prediction & Canadian & Kama & Rosa  &  \\
Canadian & \textbf{36} & 1 & 0 &  \\
Kama & 4 & \textbf{30} & 1 &  \\
Rosa & 0 & 0 & \textbf{34} &  \\
&&&& 
\end{tabular}
\caption{\textit{macierz pomyłek dla 20 neuronów w warstwie ukrytej}}
\label{tab:7}
\end{table}
\begin{table}[H]
\begin{tabular}{ccccc}
prediction & TPR (True Positive Rate) &  \\
Canadian & 36/40 (90\%) & \\
Kama & 30/31 (ok. 97\%) & \\
Rosa & 34/34 (100\%) & \\
&&
\end{tabular}
\caption{\textit{True Positive Rate dla 20 neuronów w warstwie ukrytej}}
\label{tab:8}
\end{table}
\textbf{Recognition Rate: 0.952381} \\

\textbf{Liczba neuronów: 25}\\
\linia\\

\begin{table}[H]
\begin{tabular}{ccccc}
prediction & Canadian & Kama & Rosa  &  \\
Canadian & \textbf{35} & 1 & 0 &  \\
Kama & 5 & \textbf{29} & 1 &  \\
Rosa & 0 & 1 & \textbf{34} &  \\
&&&& 
\end{tabular}
\caption{\textit{macierz pomyłek dla 25 neuronów w warstwie ukrytej}}
\label{tab:9}
\end{table}
\begin{table}[H]
\begin{tabular}{ccccc}
prediction & TPR (True Positive Rate) &  \\
Canadian & 35/40 (87,5\%) & \\
Kama & 29/31 (ok. 93,5\%) & \\
Rosa & 34/34 (100\%) & \\
&&
\end{tabular}
\caption{\textit{True Positive Rate dla 25 neuronów w warstwie ukrytej}}
\label{tab:10}
\end{table}
\textbf{Recognition Rate: 0.9333333}

\linia\\

\section*{Wnioski}
Czas trenowania wzrastał wraz z ilością neuronów w warstwie ukrytej, a malał wraz ze wzrostem parametru \textbf{\emph{threshold}}.\\

Wyniki sklasyfikowania dla każdej próby okazały się wysokie i stosunkowo bardzo do siebie zbliżone.\\

Najwyższy procent prawidłowo sklasyfikowanych obiektów został osiągnięty przy liczbie 15 i 20 neuronów w warstwie ukrytej (ok. 95\%).\\

Procent prawidłowo sklasyfikowanych obiektów wzrastał do liczby 15 neuronów, spadek został zauważony na etapie 25 neuronów.\\

Procent prawidłowo sklasyfikowanych obiektów osiągnięty przy liczbie 25 neuronów był niższy niż w przy liczbie 5, 10, 15 i 20 neuronów.\\

Na podstawie osiągniętych wyników wnioskuję, że ilość neuronów w wybranym przeze mnie pakiecie danych nie ma znaczącego wpływu na osiągane wyniki w klasyfikacji obiektów.

\section*{Oryginalne wyniki z konsoli IDE RStudio}
\linia\\
\begin{lstlisting}[language=R]
[1] "Neural network number 1 - number of neurals:  5"
[1] "Confusion matrix:"
          
prediction Canadian Kama Rosa
  Canadian       36    1    0
  Kama            4   30    1
  Rosa            0    0   33
[1] "Recognition rate:"
[1] 0.9428571

[1] "Neural network number 2 - number of neurals:  10"
[1] "Confusion matrix:"
          
prediction Canadian Kama Rosa
  Canadian       35    1    0
  Kama            5   30    0
  Rosa            0    0   34
[1] "Recognition rate:"
[1] 0.9428571

[1] "Neural network number 3 - number of neurals:  15"
[1] "Confusion matrix:"
          
prediction Canadian Kama Rosa
  Canadian       36    1    0
  Kama            4   30    0
  Rosa            0    0   34
[1] "Recognition rate:"
[1] 0.952381

[1] "Neural network number 4 - number of neurals:  20"
[1] "Confusion matrix:"
          
prediction Canadian Kama Rosa
  Canadian       36    1    0
  Kama            4   30    0
  Rosa            0    0   34
[1] "Recognition rate:"
[1] 0.952381

[1] "Neural network number 5 - number of neurals:  25"
[1] "Confusion matrix:"
          
prediction Canadian Kama Rosa
  Canadian       35    1    0
  Kama            5   29    0
  Rosa            0    1   34
[1] "Recognition rate:"
[1] 0.9333333
\end{lstlisting}

\end{document}